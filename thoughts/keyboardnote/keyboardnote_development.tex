\documentclass{article}

\usepackage{amsmath}
\usepackage{amssymb}
\usepackage{listings}

\usepackage{pgfplots}

\begin{document}

\title{TONES KeyboardPitch Model, KeyboardNote, and KeyboardNoteGroup}

\author{anematode}

\maketitle

\section{Who is this intended for?}

It's intended for me. I might read this in a few months because I didn't know what the hell I was doing. Thus, it's written in a somewhat understandable way, but it won't try to explain ideas that aren't a product of this project.

\section{KeyboardPitch}

A keyboard pitch refers to a unique note ``on the keyboard.'' It does not refer to a specific pitch; this is important because then alternative/ microtonal tuning systems would be hard to implement and that would suck. Instead, it simply refers to something like ``A4,'' ``F4,'' etc.

We model a keyboard pitch as an integer from 0 to 144 inclusive, corresponding with the notes C-1 to C11. In the old TONES it was a special class, but this is such a simple egg that I don't think it deserves another class.

When we convert a note to a "name" like C1 or Db5, we can use the following formula:

noteToName(note) $:=octaves_{note\mod 12} + stringify(\lfloor note / 12\rfloor - 1)$

where octaves is [``C''', ``C\#''', ``D'', ``D\#'', ``E'', ``F'', ``F\#'', ``G'', ``G\#'', ``A'', ``A\#'', ``B''] and zero-indexed. For example, 69 gives us index 9 and $\lfloor 5.75 \rfloor - 1 = 4$, or A4.

The inverse of this function is the nameToNote function, which is a bit more complicated to deal with flats (b), double flats (bb), sharps (\# or s), and double sharps (\#\# or ss). We first apply the regex

$\widehat{}\:([ABCDEFG])(\#|\#\#|B|BB|S|SS)?(-)?([0-9]+)\$$

to an uppercased version of the input. The first (zero-indexed) group (letter name) we reference Dict 1 to get a semitone offset from the octave ($l$). The second group (accidental) we reference Dict 2 to get a semitone offset of the accidental ($a$). The third group we reference to get whether the octave number is negative as a true or false ($b$). Finally, the fourth group is parsed to an integer as the octave number itself ($o$). Then the note's value is

nameToNote(name) $:=l + a + (b\: ? -12 : 12)\cdot o + 12.$

The relevant dicts:

Dict 1: letter\_nums = $\{ ``C": 0, ``D": 2, ``E": 4, ``F": 5, ``G": 7, ``A": 9, ``B": 11 \}$;
Dict 2: accidental\_offsets = $\{ ``\#": 1, ``\#\#": 2, ``B": -1, ``BB": -2, ``b": -1, ``bb": -2, ``s": 1, ``ss": 2, ``S": 1, ``SS": 2 \}$;

For utility and pitch estimation, there's the noteTo12TET function which converts a KeyboardPitch to its 12-TET frequency:

noteTo12TET(n, a4 = 440) $:= a4 \cdot 2^{(n - 69)/12}.$

There's also the concept of a KeyboardInterval, which is just a distance between two KeyboardPitches. Obviously, the KeyboardInterval between $a$ and $b$ is just $|a-b|$. For working with this there are the functions nameToInterval and intervalToName, but I don't expect them to be used and they are just an artifact from the old TONES, so I won't explain how they work here. Their algorithm is lengthy but straightforward. To get the value in cents of a KeyboardInterval \textbf{in 12 TET}, multiply by 100.

\end{document}